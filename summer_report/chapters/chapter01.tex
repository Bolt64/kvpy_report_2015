\section{Some preliminary notions}

\begin{defn}
A group $(G, \circ)$ is a set $G$ along with an associated binary operation $\circ$ called multiplication, such that it satisfies the following axioms:
\begin{itemize}
\item If $a$ and $b$ belong to $G$, then $a \circ b$ also belongs to $G$. (Closure axiom)
\item Multiplication is associative, i.e. if $a$, $b$ and $c$ belong to $G$, then $a \circ (b \circ c) = (a \circ b) \circ c$. (Associativity axiom)
\item There exists an element $e$ in the group, called identity, such that for all $a \in G$, $a \circ e = e \circ a = a$. (Existence of identity)
\item For every element $a \in G$, there exists an element $b$, called the inverse of $a$, such that $a \circ b = b \circ a = e$ (Existence of inverse)
\end{itemize}
\end{defn}

\begin{defn}
A subgroup $(S, \circ)$ of a group $(G, \circ)$, is a subset $S$ of the set $G$, which is a group under the operation $\circ$.
\end{defn}

It trivially follows from the axioms that the identity element is unique, and an element has a unique inverse. Now for some examples:

\begin{exmp}
\label{int}
The set of integers $\mathbb{Z}$ under the operation of addition is a group.
\end{exmp}

\begin{exmp}
\label{int_mod}
The set of integers $\mathbb{Z}$ under the operation of addition modulo some positive integer $n$ is a group. It is denoted by the symbol $\mathbb{Z}/n\mathbb{Z}$
\end{exmp}

\begin{defn}
A homomorphism is a mapping $\phi$ from a group $(G, \circ)$ to another group $(G', *)$ such that it preserves the group structure, or in other words, for any $a,b \in G$, $\phi(a \circ b) = \phi(a) * \phi(b)$.
\end{defn}

\begin{exmp}
Consider the mapping $\phi$ from $\mathbb{Z}$ to $\mathbb{Z}/7\mathbb{Z}$, such that $\phi(x) = x (\mod 7)$. Clearly, this is a homomorphism.
\end{exmp}

\begin{defn}
An isomorphism between $G$ and $G'$ is a homomorphism which is one-one and onto. If an isomorphism exists between two groups, they are said to be isomorphic.
\end{defn}

For all practical purposes, two isomorphic groups are same in every aspect, so it's common to treat isomorphic groups as equivalent.

\section{Presentations}

One of the problems one immediately faces is how to represent groups. The groups mentioned in the example were nice, in the sense that they deal with sets that are familiar to most people. However, for any arbitrary group, representing them compactly might be a problem. One way of representing a group, at least a finite group is to write out its multiplication table, that is, for every possible ordered pair, write out their product. This method works, but only for finite groups, and even this grows cumbersome after the group exceeds a certain size.

One way to improve upon this is to note that a multiplication table has a lot of redundant information, information that can be deduced by just using the algebraic structure of the group. For example, if an entry in the table is of the form $a^3 \circ a^4 = b$, then one can immediately deduce from the associativity axiom that $a^2 \circ a^5 = b$, $a \circ a^6 = b$ and so on. Before we can try and answer that question, we'll need to define a few terms. \cite{johnson}

\begin{defn}
A subset $S$ of a group $(G, \circ)$ is said to generate $G$ if every member of $G$ can be written as the product of powers of elements of $S$.
\end{defn}

Notice that in both examples, Example \autoref{int} and Example \autoref{int_mod}, the set $S = \{1\}$ generates the whole group. But in the second example, there is an additional condition that $1^n=0$ which is not there in the first example. Now consider a singleton set $T = \{a\}$ and the group $H$ generated by it. Since $H$ is generated by $T$, every element of $H$ is of the form $a^p$ for some integer $p$. Now, if $H$ is finite, we can, by Pigeonhole principle, find a non negative $q$ such that $a^q=e$. Otherwise, $H$ is infinite and isomorphic to $\mathbb{Z}$. Here's a formal proof:

\begin{thm}
\label{class1}
If $H$ is a group generated by the singleton set $S = \{a\}$, then $H$ is either isomorphic to $\mathbb{Z}$ or $\mathbb{Z}/q\mathbb{Z}$, where $q$ is a nonnegative integer.
\end{thm}

\begin{proof}
We can split the proof into two cases, when $H$ is finite, and when it's infinite.
\begin{enumerate}
\item If $H$ is finite, there exists a positive integer (it can't be $0$, because a group has at least one element) $q$, which is the size or the order of the group. Consider the sequence $A = \{a^0, a^1, a^2 \cdots , a^q\}$. Since we have $q+1$ elements, but the group only has $q$ distinct objects, by the pigeonhole principle, at least two objects must be identical. Let them be $a^m$ and $a^n$ ($m>n$). Let $p=m-n$ (it follows that $q \geq p$). Then, $a^p=e$. And consequently $a^l = a^{l \mod p}$. Since, we can have at most $p$ distinct elements modulo $p$, the group can have at most $p$ elements. This implies $p \geq q$. And following from the last inequality, we get $p=q$. From this we conclude that $a^q=e$, and for all positive integer $l$ less than $q$, $a^l \neq e$. Now we can construct an isomorphism with $\mathbb{Z}/q\mathbb{Z}$. We can now confidently say that there is a unique way of writing every element of $H$ in the form $a^l$, where $0 \leq l < q$. Let the mapping be $f(a^l) = l$ where $l$ varies from $0$ to $q-1$. Since, the kernel of $f$ is just $\{e\}$ and the range is all of $\mathbb{Z}/q\mathbb{Z}$, $f$ is an isomorphism. Hence, if $H$ is finite, it is isomorphic to $\mathbb{Z}/q\mathbb{Z}$.

\item If $H$ is infinite, there can't exist any positive integer $q$ such that $a^q=e$, otherwise the group would have at most $q$ elements, and that would mean its order was finite. This also implies that if $a^m = a^n$, then $n=m$. This means that there is a unique way of writing every element of $H$ in the form $a^l$, where $l \in \mathbb{Z}$. Consider the mapping $f$ such that $f(a^l)=l$. Clearly, this is onto on $\mathbb{Z}$ and it's also one-one. Hence $f$ is an isomorphism. Ergo, $H$ is isomorphic to $\mathbb{Z}$ if $H$ is infinite.
\end{enumerate}
Hence, we have managed to classify all the groups generated by a singleton set up to isomorphism.
\end{proof}

We have managed to make a little progress on our goal of compressing information about a group. For example, the group $\mathbb{Z}/q\mathbb{Z}$ is completely represented by the information that it is generated by a singleton set and it is of the order $q$. Let's introduce a new notation to represent this information. We can represent the group in the following manner:
$$\mathbb{Z}/q\mathbb{Z} = \langle a;a^q=e \rangle$$
Further on, we can get rid of the trailing ``$=e$" and just write
$$\mathbb{Z}/q\mathbb{Z} = \langle a;a^q \rangle$$
We can use the same notation to represent the group $\mathbb{Z}$, writing it down as
$$\mathbb{Z} = \langle a;\rangle$$
The notation we used above is called a presentation, with the terms to the left of the semi colon called generators, and the terms to the right are called relations. The generators, as the name suggests, generate the group, or in other words, every element of the group can be written as the product of powers of the generators. There is also no restriction on the generators to be necessarily distinct, so referring to a `set' of generators is an abuse of terminology. The relators consist of \emph{words} which evaluate to identity in the group. 
\begin{defn}
Given a set $S$ called the generating set, a word is a finite (possibly empty) sequence $a_{i}^{j}$, where $a_i \in S$, and $j = \pm 1$. A word is reduced if $a_i^{j}$ is never followed by $a_i^{-j}$.
\end{defn}
\begin{exmp}
Given a generator set $S = \{a,b\}$, $aba^{-1}b^{-1}$ is a reduced word, but $abb^{-1}a^{-1}$ is not a reduced word.
\end{exmp}

If a word is not reduced it's possible to reduce it simply by iterating over the `letters' in the word, and at the first occurrence of a letter being followed by its inverse letter, deleting both the letters from the word, and starting the process all over again until one has a reduced word. Two words are considered equivalent if they can be converted from one to the other just by either deleting adjacent inverse letters, or inserting pairs of adjacent inverse letters somewhere in the word. One can define the operation of multiplication on words as well. The product of two words is just concatenation of their sequence of letters together. And similarly, one defines the formal inverse of a word as the sequence of letters of the word taken in reverse, and the exponent on them inverted.

\begin{lem}
Inserting or deleting adjacent pair of inverse letters does not change the reduced form of the word.
\end{lem}

\begin{proof}
The proof has two parts: showing the statement is true for insertions, and the showing it's true for deletions:
\begin{description}
\item[Insertion] Let the pair of inverse letters $aa^{-1}$ be inserted somewhere in the word. The new word can be broken up in 3 subwords, the subword preceding the insertion point, the subword $aa^{-1}$ and the subword following the insertion point. The preceding subword will get reduced in the same manner as it did in the original word, with two possible cases arising: either the last letter of the subword is $a^{-1}$; in that case, that will get reduced with the first $a$ of the second subword, and the $a^{-1}$ of the second subword will get concatenated to the first reduced subword. Note that it can't reduced any further, otherwise the first reduced subword would have not have $a^{-1}$ in the trailing position. Finally, the last subword will get reduced like it did in the original word, and we'll get the original reduced word back. In the other case where the first subword does not end with $a^{-1}$, the middle subword will get reduced to the empty word and we'll get back the original word again.
\item[Deletion] Assume the pair to get deleted to be $aa^{-1}$. Split the shorter word into two subwords, the one preceding the deletion point and the one following. Two cases can arise again, either the preceding word had a trailing $a^{-1}$, or not. If it didn't the deleted pair $aa^{-1}$ would have gotten deleted anyway during the reduction, so the reduced word remains the same. If the preceding word did have a trailing $a^{-1}$, then that would get reduced with the leading $a$ of the deleted subword, leaving a trailing $a^{-1}$, and that would lead to the same reduced word. 
\end{description}
Hence, inserting or deleting adjacent pairs of inverse letters does not change the reduced form.
\end{proof}

\begin{exmp}
Let $W_1$ be the word $ab$ and $W_2$ be the word $b^{-1}a$. The product word $W_1W_2$ is $abb^{-1}a$ and $W_1^{-1}$ is $b^{-1}a^{-1}$.
\end{exmp}
And since we have defined an equivalence relation on the words, we can partition the set of words $W$ generated by a set $S$, we can partition $W$ into equivalence classes. Additionally, a group structure can be imposed on the partition classes as we'll do in \autoref{freegroups}.

The relators in a presentation therefore determine how the various generators interact with each other. For example, if $aba^{-1}b^{-1}$ was one of the relators in a group generated by just $\{a,b\}$, that would imply $ab=ba$, or the group was abelian. But given our current definition of a presentation as a tuple of generators and relators, it can be shown that most presentations do not specify a unique group (up to isomorphism). We need a more rigorous notion of what a presentation is.

\section{Free Groups}\label{freegroups}
Before we formalize the notion of a presentation of a group, we need to define free groups. There are many ways of defining or constructing free groups, from formal algebraic methods, to more geometric methods using infinite trees\cite{meier}. The technique we'll use here to define free groups imitates how fundamental groups of topological spaces are defined. Before we define a free group, here's a lemma we'll need.

\begin{lem}\label{well_def}
If two words $a_1$ and $a_2$ are equivalent, and the words $b_1$ and $b_2$ are equivalent, then $a_1b_1$ is equivalent to $a_2b_2$.
\end{lem}

\begin{proof}
Let $S$ be the generating set of $a_1$, $a_2$, $b_1$ and $b_2$. If $c$ is an element of $S$, then inserting $cc^{-1}$ or $c^{-1}c$ somewhere in a word does not change its equivalence class. Conversely, deleting a pair of adjacent inverse letters does not change the equivalence class of the word either. Also, these two operations are reversible and are each other's inverse. Saying that $a_1$ and $a_2$ are equivalent means that they reduce to the same reduced word, which means there's a sequence of finite moves consisting of all the deletions that lead from $a_1$ to $a_{reduced}$ and then a finite sequence of insertions, which are the inverses of the deletions that lead from $a_2$ to $a_{reduced}$. Similarly, there exist a finite sequence of moves that leads from $b_1$ to $b_2$. Applying those moves on $a_1b_1$, we reach $a_2b_2$, which shows $a_1b_1 \equiv a_2b_2$.
\end{proof}

\begin{defn}
A free group on the set $S$ is defined to be the set of equivalence classes on the set of words generated by the set $S$, with identity being the equivalence class of the empty word, and group multiplication of classes $A$ and $B$ being the equivalence class of the product $ab$, where $a \in A$ and $b \in B$. In other words:
$$A \circ B = [a \cdot b]$$
And inverse of a class $A$ is the equivalence class of the inverse of $a$, where $a \in A$.
\end{defn}
By lemma \autoref{well_def}, the multiplication and inverse are well defined.

Using just this definition, one can go about proving properties of free groups. Here's a simple example of such a property:

\begin{prop}\label{double}
If $\alpha$ is an element of a free group $G$ and $\alpha^2=e$, where $e$ is the identity, then $\alpha=e$.
\end{prop}

\begin{proof}
Let $a$ be the reduced word belonging to the equivalence class $\alpha$. $aa$ then belongs to the equivalence class $\alpha^2$. But $\alpha^2=e$, which means that $aa$ is equivalent to the empty word. Since $a$ is reduced, but $aa$ reduces to the empty word, the only place reduction can happen is between the end of the first $a$ and the beginning of the second $a$. Now assume that $a$ is non empty. This means $a[m] = a[n-m-1]$, where $a[m]$ represents the $m^{th}$ letter of $a$ (starting from $0$) and $n$ is the length of $a$. Now $a$ could either have an even number of letters in which case $a[\frac{n}{2} -1]$ letter is the inverse of the adjacent $a[\frac{n}{2}]$ letter and those two cancel each other out. This contradicts the fact that $a$ is reduced. Ergo, $a$ must be empty. And if $a$ has an odd number of letters, then the middle letter must be its own inverse, which is not true for any letter in the basis. Hence $a$ can only have an even number of letters and in that case, it's the identity. This completes the proof that $\alpha=e$
\end{proof}

Using this result, we can prove a slightly stronger result, which is that no power of a non trivial element in a free group is ever trivial.

\begin{prop}
If $\alpha$ is an element of a free group $G$ and $\alpha^p=e$, where $e$ is the identity and $p$ is a positive integer, then $\alpha=e$.
\end{prop}

\begin{proof}
We use the same trick we used in the last proof. Let $a$ be the reduced word from $\alpha$. Then $a^p$ reduces to the empty word, but reduction can only happen where one word is joined to the next, since the word $a$ is reduced. Proceeding by the same arguments as in the last proof, we show that $a$ necessarily must be the empty word, and that concludes the proof. 
\end{proof}

These theorems also enable us to tell whether something is a free group or not. If some element has a finite order in a group, that group is certainly not a free group, but the converse is not true always. An example without a proof of the latter is the group $\mathbb{Z} \times \mathbb{Z}$. In this group, every element has an infinite order, but the group is not free.

\subsection{On the isomorphisms of free groups\cite{lyndon}}
\begin{thm}\label{rank}
If the bases of two free groups are isomorphic (i.e. have the same cardinality) then the groups are isomorphic.
\end{thm}

\begin{proof}
Consider a bijective mapping $f$ between the bases $B_1$ and $B_2$ of groups $X$ and $Y$. Extend $f$ to a homomorphism in the following manner:
$$f(x_1^{e_1}x_2^{e_2} \cdots x_n^{e_n}) = f(x_1)^{e_1}\cdots f(x_n)^{e_n}$$
where $x_i$s belong to the basis $B_1$. This, by definition is a homomorphism. All we need to show is that it's injective and surjective. Since the whole basis $B_2$ lies in the image of $f$, the homomorphism certainly is surjective. Now all we need to show if that if $f(\alpha)=\mathrm{identity}_{Y}$, then $\alpha=\mathrm{identity}_{X}$. Let $x$ be a word in $\alpha$, then $f(x)$ is a word in $f(\alpha)$. For each reduction we make in $f(x)$, we can make the corresponding reduction in $x$, hence if $f(x)$ reduces to the empty word, so does $x$, which shows the kernel is trivial and $f$ is an isomorphism.
\end{proof}

The converse, although true, is a little trickier to prove, and little more machinery needs to be developed before it can be proven. See theorem \autoref{conv}.

\section{A digression to quotient groups}
Before we proceed any further, we'll need a little more `machinery', so to speak. First, we need the notion of a coset of a subgroup.

\subsection{Cosets and quotient groups}

\begin{defn}
Given a group $G$, a subset $H$, and an element $a \in G$, the left coset $aH$ is defined as the set
$$aH = \{ah\ |\ h \in H\}$$
A right coset $Ha$ is defined in a similar manner.
\end{defn}

\begin{defn}
A subgroup $H$ of a group is called normal \emph{iff} for all $a \in G$, $Ha = aH$. 
\end{defn}

\begin{defn}
A cartesian product of two subsets $A$ and $B$ of a group $G$ is the set
$$A \times B = \{ab\ |\ a \in A,\ b \in B\}$$
\end{defn}

This is where the convenient properties of normal subgroups come in. Since the smaller term can commute around the larger subgroup, so to speak, cosets of normal subgroups multiply very nicely.

\begin{exmp}
If $H$ is a normal subgroup of $G$ and $a$ and $b$ belong to $G$, then 
$$aH \times bH = Ha \times bH = Hab \times H = abH \times H = ab(H \times H)$$
But $H \times H$ is just $H$ again, since $H$ is a group. Hence
$$aH \times bH = abH$$
\end{exmp}

This example shows that cosets of normal groups multiply nicely and this lend themselves naturally to a group structure.

\begin{defn}
Given a group $G$ and a normal subgroup $H$, we can define the quotient group $G/H$ whose elements are the cosets of $H$ in $G$, the group multiplication is defined as the cartesian product of cosets. The above example shows the multiplication is associative. The identity element is the set $H$ itself, and the inverse of $aH$ is $a^{-1}H$.
\end{defn}

\subsection{Some useful theorems\cite{herstein}}

\begin{thm}
That $H$ is a normal subgroup of $G$ is equivalent to saying that $aHa^{-1}=H$ for all $a \in G$ where the set $aHa^{-1}$ defined as 
$$aHa^{-1}=\{aha^{-1}\ |\ h \in H\}$$
\end{thm}

\begin{proof}
If $H$ is normal in $G$, then for any $a \in G$, $aH=Ha$. Since $aH=Ha$, $(aH)a^{-1}$ is equal to the set $(Ha)a^{-1}$ which reduces to $H$. This shows that if $H$ is normal in $G$, $aHa^{-1}=H$.

Now for the converse. If $aHa^{-1}=H$, then $(aHa^{-1})a = Ha$, but $(aHa^{-1})a = aH$, hence $aH = Ha$.
\end{proof}

\begin{thm}\label{normal}
The intersection of two normal subgroups $A$ and $B$ of $G$ is normal in $G$.
\end{thm}

\begin{proof}
Let $C = A \cap B$. Clearly $aCa^{-1}$ is a subset of both $A$ and $B$ for all $a \in G$. Hence $aCa^{-1} \subset C$. But if that's true for all $a \in G$, that's true for $a^{-1}$ as well. That means $a^{-1}Ca \subset C$ or $C \subset aCa^{-1}$. Hence $aCa^{-1} = C$ for all $a \in G$ and $C$ is normal. This proof in fact works for any arbitrary number of normal subgroups, even infinitely many.
\end{proof}

\begin{thm}
The kernel of a homomorphism $f$ from $G$ to $G'$ is a normal subgroup of $G$.
\end{thm}

\begin{proof}
Let the kernel of $f$ be $H$. If we show that $aHa^{-1}=H$ for all $a \in G$, we are done. Consider any $k \in aHa^{-1}$. $k$ is of the form $aha^{-1}$ for some $h \in H$. But $f(k)= f(a) \cdot f(h) \cdot f(a^{-1}) = e$. This shows $k$ is in the kernel of $f$, hence $aHa^{-1} \subset H$. But this is true for any $a \in G$. In particular, take $a^{-1}$. $a^{-1}Ha \subset H$, but that means $H \subset aHa^{-1}$, which proves $aHa^{-1}=H$. This shows the kernel is a normal subgroup.
\end{proof}

\begin{thm}[First Isomorphism Theorem]
If $f$ is a homomorphism from $G$, then the image of $f$, $I$ is isomorphic to $G/\mathrm{Ker}(f)$.
\end{thm}

\begin{proof}
Let $Q$ represent the quotient group $G/\mathrm{Ker}(f)$ and $H$ the kernel of $f$. Elements of $Q$ are of the form $aH$, where $a$ belongs to $G$. Consider the mapping $\phi$ which maps from $Q$ to $I$ in the following manner: $\phi(aH) = f(a)$. First of all, we need to show that $\phi$ is well defined. Assume that $aH=bH$, $a$ not necessarily distinct from $b$. Clearly, $ah_1=bh_2$ for some $h_1$ and $h_2$ in $H$. $f(ah_1)=f(bh_2)$, but $h_1$ and $h_2$ lie in the kernel, so $f(a)=f(b)$. This shows that the value of $\phi$ only depends upon the element of $Q$ we take, and not the way we represent it. Also, since $H$ is normal, $aH \times bH=abH$. Hence $\phi(aH \times bH) = \phi(ab) = \phi(aH)\phi(bH)$. This shows $\phi$ is a homomorphism. Now to show $\phi$ is surjective, let $i$ be some element of $I$. There exists then some $a \in G$, such that $f(a)=i$. And therefore, $\phi(aH)=i$, which shows that $\phi$ is surjective. And if $f(a)=e$, then $a \in H$. But $aH=H$, which is the identity element of $Q$. This shows $\phi$ is injective. Ergo, $\phi$ must be an isomorphism.
\end{proof}



\section{Presentations revisited}

By theorem \autoref{rank}, all free groups generated by a basis of size $n$ are isomorphic, so it makes sense to talk of all them as one, as the free group of rank $n$, or $\mathbb{F}_n$. 

\begin{lem}\cite{meier}\label{quotient}
If $G$ is a group generated by $n$ elements, then $G$ is isomorphic to a quotient group of $\mathbb{F}_n$
\end{lem}

\begin{proof}
Let $B= \{b_1, b_2, \cdots , b_n\}$ be the basis of $\mathbb{F}_n$. Let $S=\{a_1, a_2, \cdots, a_n\}$ be the generating set of $G$. Consider the mapping $f(b_i)=a_i$. Extend this mapping linearly to all of $\mathbb{F}_n$. Since the set $S$ generates $G$, $S$ is the image of $f$. Now, we have a homomorphism from $\mathbb{F}_n$ to $G$, so we can use the first isomorphism theorem and claim that $G \approx \mathbb{F}_n/\mathrm{Ker}(f)$. 
\end{proof}

\begin{lem}\cite{meier}
If $R$ is a subset of the group $G$, then there exists a normal subgroup $H$, called the normal closure of $R$ such that if two normal subgroups $A$ and $B$ contain $R$, then they contain $H$.
\end{lem}

\begin{proof}
There exists at least one normal subgroup that contains $R$, that is the group $G$ itself. Consider the set $S$ of all normal subgroups of $G$ containing $R$. By the extension of theorem \autoref{normal}, they intersection of all members of $S$ must be normal in $G$, and that subgroup also contains $R$. Furthermore, it satisfies the property the theorem asks for, hence we're done.
\end{proof}

We are now in a position to define the presentation of a group.

\begin{defn}
A presentation $P$ is a tuple of a set $S$, and a set of reduced words $R$ from $\mathbb{F}_S$, also called relations. A presentation is represented in the following manner: $\langle S;R\rangle$.
\end{defn}

\begin{defn}
A presentation $\langle S;R\rangle$, generates a quotient group $Q$, which is precisely the group $\mathbb{F}_S/N$, where $N$ is the normal closure of $R$.
\end{defn}

\begin{defn}
A group $G$ is said to have a presentation $\langle S;R \rangle$ if it is isomorphic to the quotient group generated by the presentation.
\end{defn}

One question still remains, whether a group actually has a presentation; to be more precise, is every group isomorphic to some quotient of a free group, and that is answered by lemma \autoref{quotient}. 

With that, we have precisely defined what we mean by a presentation. Here are some concrete examples of presentations of groups.

\begin{exmp}
The group $\mathbb{Z}$ can be presented in the following manner: $\langle a;\rangle$. If no relations are specified, the relation set is taken to be the empty one. Hence, the smallest normal subgroup containing the empty set is just the trivial subgroup $\{e\}$, and the quotient group is just $\mathbb{F}_1/\{e\}$, which is isomorphic to $\mathbb{F}_1$, which was shown before to be isomorphic to $\mathbb{Z}$.
\end{exmp}

\begin{exmp}
The group $\mathbb{Z} \times \mathbb{Z}$ is presented in the following manner: $\langle a,b;aba^{-1}b^{-1}\rangle$. Let $N$ be the smallest normal subgroup of $\mathbb{F}_2$ containing $aba^{-1}b^{-1}$. And let $Q$ be the quotient group $\mathbb{F}_2/N$. The elements of $Q$, or the cosets of $N$ are distinguished by a tuple of integers representing the sum of exponents of $a$ and $b$ respectively. To put it in simpler terms, $ab$ and $a^2ba^{-1}$ both belong to the same coset of $N$ because they both have the sum of coefficients of $a$ and $b$ respectively $1$ and $1$. So an element of $Q$ is identified by a tuple of integers, which is the same as $\mathbb{Z} \times \mathbb{Z}$.
\end{exmp}

\section{Transformations}

\begin{lem}\label{elementary}
Given a group $G$, and an ordered set of elements of $G$, $\{a_i\}$, then the smallest normal group containing $\{a_i\}$ does not change if $a_i$ is replaced by $a_i^{-1}$, $a_i a_j$, or $a_j a_i$, where $j \neq i$.
\end{lem}

\begin{proof}
To prove the theorem, it's sufficient to show that every normal group containing $\{a_i\}$ also contains the modified $\{a_i\}$ and vice versa. Let $N$ be a normal subgroup containing $\{a_i\}$. Since it's a group, it also contains $a_i^{-1}$ for all $i$. And the same argument holds true if $N$ is a normal subgroup containing the modified $\{a_i\}$. Thus, the normal subgroups are invariant under the inversion transformation.\\
Now suppose $N$ is a normal subgroup containing $\{a_i\}$. Since it's a group, it must also contain $a_ia_j$ for every $ij$ pair. Now assume the converse. Assume $N$ is a normal subgroup containing the modified $\{a_ia_j\}$, where $a_i$ is replaced by $a_ia_j$. But $N$ still contains $a_j$, and by extension $a_j^{-1}$, so the product of $a_ia_j$ and $a_j^{-1}$, which is $a_i$, must also be in the group. Hence, the normal subgroups are invariant under right multiplication. The same proof also holds for left multiplication.
\end{proof}

\begin{lem}\label{conj}
Given a group $G$, and an ordered set of elements of $G$, $\{a_i\}$, then the normal subgroups of $G$ are invariant under the replacement of $a_i$ by $xa_ix^{-1}$, where $x \in G$.
\end{lem}

\begin{proof}
Consider a normal subgroup $N$ containing $\{a_i\}$. Since it's a normal subgroup, $xNx^{-1}=N$, hence $N$ contains $xa_ix^{-1}$ for any $i$. Now consider a normal subgroup containing the modified $\{a_i\}$, where $a_i$ has been replaced with $xa_ix^{-1}$. Since $N$ is normal $x^{-1}Nx=N$, so $N$ contains $a_i$. Hence normal subgroups are invariant under conjugation by generators.
\end{proof}

The transformations mentioned in lemma \autoref{elementary} are called Nielsen transformations, and the transformation mentioned in lemma \autoref{conj} is called the conjugation transformation.

\begin{lem}
If $\langle S;R \rangle$ presents a group $G$, then so does $\langle S; R' \rangle$, where $R'$ is $R$ transformed by the elementary Nielsen transformations or the conjugation transformation.
\end{lem}

\begin{proof}
Since the smallest normal subgroup containing $R$ and $R'$ is the same, by lemmas \autoref{elementary} and \autoref{conj}, the quotient group remains the same and so does the group.
\end{proof}

\begin{defn}
A group is said to be finitely presented if it can be presented in the form $\langle G;R\rangle$, where both $G$ and $R$ are finite sets.
\end{defn}

\begin{defn}
A finite presentation, i.e. $G$ and $R$ are finite, is said to be balanced if $|G|=|R|$.
\end{defn}

It is obvious the if one takes the presentation $\langle S;S \rangle$, the presentation generates just the trivial group, because the normal closure of $S$ in $\mathbb{F}_S$ will be the whole group, which means the quotient reduces to $\{e\}$. This kind of presentation is also called the trivial presentation. It's obvious that for a finite $S$, a trivial presentation is balanced. It's also obvious that the elementary Nielsen transformations and the conjugation transformation does not change the cardinality of the set of relations, i.e. these transformation preserve balance. It's only a natural question to ask now whether we can obtain any finite balanced presentation of the trivial group starting from the trivial presentation. Here's a more formal statement:

\begin{conj}[Andrews-Curtis Conjecture]
A finite balanced presentation of the trivial group can be transformed into the trivial presentation using a finite sequence of elementary Nielsen transformations and the conjugation transformation.
\end{conj}

The Andrew-Curtis conjecture as of yet, is an open problem. It's been shown to be true under stronger conditions like limiting the size of relations and generators\cite{genetic}, but the most general statement is unproven. It's also known to be false under weaker conditions, i.e. when one just deals with general groups and not trivial groups in particular.